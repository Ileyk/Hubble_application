%%%%%%%%%%%%%%%%%  Debut du fichier Latex  %%%%%%%%%%%%%%%%%%%%%%%%%%%%%%
\documentclass[letterpaper,12pt,onecolumn]{article}

%%% Pour un texte en francais

%%\usepackage[applemac]{inputenc}
%\usepackage[francais]{babel}
	         % encodage des lettres accentuees
\usepackage[T1]{fontenc}
\usepackage[utf8]{inputenc}          % encodage des lettres accentuees
%\usepackage{graphicx}
%%\usepackage{graphicx} \def\BIB{}
\usepackage[paper=a4paper,textwidth=140mm,left=2.1cm,right=2.1cm,top=2.7cm,bottom=2.7cm]{geometry}
\usepackage{multicol}
\usepackage{graphicx,wrapfig,lipsum} 
%\def\BIB{}
\usepackage[font=footnotesize]{caption}
\usepackage{subcaption}
\usepackage[pdftex]{hyperref}
\usepackage{natbib}
\usepackage{url}
\usepackage{perpage} %the perpage package
\MakePerPage{footnote} %the perpage package command
\hypersetup{
    colorlinks,%
    citecolor=black,%
    filecolor=black,%
    linkcolor=black,%
    urlcolor=blue     % can put red here to visualize the links
}


\DeclareUnicodeCharacter{00A0}{ }

%%% Quelques raccourcis pour la mise en page
\newcommand{\remarque}[1]{{\small \it #1}}
\newcommand{\rubrique}{\bigskip \noindent $\bullet$ }

\newcommand{\ignore}[1]{}

\renewcommand*\rmdefault{iwona}

\pagenumbering{gobble}

%\bibliographystyle{abbrvnat}
%\setcitestyle{authoryear,open={((},close={))}}

%\renewcommand{\thefootnote}{\roman{footnote}}

% -------------------------------------------------
\newcommand{\horrule}[1]{\rule{\linewidth}{#1}} % Create horizontal rule command with 1 argument of height

\title{	
\vspace*{-2.5cm}
%\normalfont \tiny 
%%\textsc{Paris Diderot} \\ [25pt] % Your university, school and/or department name(s)
%\horrule{0.5pt} \\[0.4cm] % Thin top horizontal rule
\Large Research summary \\ % The assignment title
%\horrule{2pt} \\[0.5cm] % Thick bottom horizontal rule
}
\author{\tiny} % Your name
\date{\tiny }%\normalsize\today} % Today's date or a custom date
% -------------------------------------------------

%\makeatletter
%\def\@xfootnote[#1]{%
%  \protected@xdef\@thefnmark{#1}%
%  \@footnotemark\@footnotetext}
%\makeatother

\begin{document}

%\bibpunct{[}{]}{;}{n}{,}{,}

%%%%%%%%%%%%%%%%%%%%%%%%%  PREMIERE PAGE %%%%%%%%%%%%%%%%%%%%%%%%%%%%%%
%%% DANS CETTE PAGE, ON REMPLACE LES INDICATIONS ENTRE CROCHETS [...]
%%% PAR LES INFORMATIONS DEMANDEES
%%%%%%%%%%%%%%%%%%%%%%%%%%%%%%%%%%%%%%%%%%%%%%%%%%%%%%%%%%%%%%%%%%%%%%%

\maketitle
\thispagestyle{empty}

\vspace*{-1cm}

Transfer of angular momentum : which fraction of the wind is captured? Which escapes? Influences the mutual spins and the orbital decay.
Spin of BH and NS

, especially around an accreting \ns where the disc must formed beyond the magnetosphere which truncates its inner edge

%\indent More generally, the B field opens the door to a wide range of accretion-ejection mechanisms in the disc itself (Blandford-Payne 82, Casse 04), provided we have a physically-motivated disc thickness profile (see previous section). Transition from jet emitting discs to standard accretion discs (SS75) as a possible explanation for the two states observed in LMXB (Ferreira 06).\\ \\

%\section*{Conclusion}
%
%Bedrock model 
%
%Longer term prospective

\newpage

%\newgeometry{left=2cm,right=2cm,top=2.5cm,bottom=2.5cm}
\setlength{\bibsep}{5pt}
\small
\bibliographystyle{agsm}
\bibliography{/Users/Ileyk/Documents/Bibtex/research_statement_no_url}

\newpage

\phantom{n}

\end{document}
%%%%%%%%%%%%%%%%%  Fin du fichier Latex  %%%%%%%%%%%%%%%%%%%%%%%%%%%%%%

